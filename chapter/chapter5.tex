\chapter{The quantum Fourier transform and it's application}


\Textbf{5.1} 


We can write the Fourier quantum transformation as a matrix U.
By definition, for an arbitrary ket $|x\rangle = \sum_{j=0}^{N-1} x_j | j \rangle$, the 
quantum fourier transform maps $|x\rangle$ to $|y\rangle$ as such, noting $w = e^{2 i \pi /N}$ :

$$F(|x\rangle) = |y\rangle = \frac{1}{\sqrt{N}}  \sum_{j=0}^{N-1}  \sum_{n=0}^{N-1} x_n w^{k n} | j \rangle$$

the Fourier quantum transformation can then be written as a matrix equation, giving the Fourier transform's
matrix, $F |x\rangle = |y\rangle$, with :

$$ F =  \frac{1}{\sqrt{N}} \begin{bmatrix}
	1      & 1       & \hdots & 1         \\
	1      & w       & \hdots & w^{N-1}   \\ 
	1      & w^2     & \hdots & w^{2(N-1)} \\
	\vdots & \vdots  & \vdots & \vdots     \\
	1      & w^{N-1} & \hdots & w^{(N-1)(N-1)}
\end{bmatrix}.$$

Let's now show that this matrix is indeed unitary.
If we note $A = F F^{\dagger}$, then 
$$A_{j,k} = \sum_{n=1}^N F_{j,n} F^{\dagger}_{n,k} = \frac{1}{N} (1 + w^{j-n}+w^{2(j-n)}+...+w^{(N-1)(j-n)})$$.

If $j=n$ then $A_{j,k} = \frac{N}{N} = 1$.

If $j \neq n$ then $A_{j,k} = 0$ according to the following orthogonal property over the roots of unity : 

$$ \sum_{k=0}^{N-1} (z^{jk})^\dagger z^{j' k} = 0 \quad \mathtt{if } \quad j \neq j'$$

We then see that $F F^{\dagger} = I$.
It is straightforward to show that $F F^{\dagger} =F^{\dagger} F$, we can then conclude that
the quantum Fourier transform is unitary.

\Textbf{5.2} 

the $N$ qubit state $|00...0 \rangle$ is a computational basis state, we can use directly the provided  formula 5.2 :

$$  |j\rangle \rightarrow \frac{1}{\sqrt{N}} \sum_{k=0}^{N-1}  e^{2 i \pi j k /N} |k \rangle$$

applying it to $|00...0 \rangle$ we get :

$$|00...0 \rangle \rightarrow \frac{1}{2^{N/2}} \sum_{k=0}^{2^N-1} |k \rangle$$

with $|k \rangle$ the $2^N$ computation basis states for $N$ qubits.

