%
% generate PDF file
% pdflatex solnQCQI.tex
%
\documentclass[11pt]{book}
\usepackage[utf8]{inputenc}
\usepackage{amsmath,amsfonts,amssymb,amsthm}
\usepackage[width=16.00cm, height=24.00cm]{geometry}
\usepackage[dvipdfmx]{graphicx}
\usepackage[english]{babel}

%図の場所をなるべく指定した場所にする
\usepackage{booktabs}
\usepackage{here}

\RequirePackage[l2tabu, orthodox]{nag}
\usepackage[all, warning]{onlyamsmath}

%単位を書くときに使う
\usepackage{siunitx}

\usepackage{CJKutf8}
\usepackage{ascmac} % screen
\usepackage{ulem}
\usepackage{cases}
\usepackage{braket}
\usepackage{dsfont}
\usepackage{ascmac}
\usepackage{url}
\usepackage{hyperref} % hyper link
\usepackage{ccicons} % creative commons license icon
\usepackage{fancyhdr} % footer
%\pagestyle{fancy}
%\cfoot[\href{http://creativecommons.org/licenses/by-nc-sa/4.0/}{Creative Commons Attribution-NonCommercial-ShareAlike 4.0 International License}.]{}
\usepackage{color}


\usepackage{fancyhdr}
\setlength{\headheight}{15.2pt}
\pagestyle{fancy}
\lhead[\leftmark ]{\thepage}
\rhead[\thepage]{\leftmark}

\cfoot{\footnotesize \textcopyright 2018 goropikari - \href{http://creativecommons.org/licenses/by-nc-sa/4.0/}{Creative Commons Attribution-NonCommercial-ShareAlike 4.0 International License}}

% コマンド定義
\DeclareMathOperator{\Tr}{Tr}
\newcommand{\norm}[1]{\left\lVert#1\right\rVert} % norm ||x||
\newcommand{\kb}[1]{\ket{#1}\hspace{-1mm} \bra{#1}} % |x><x|
\newcommand{\kbt}[2]{\ket{#1}\hspace{-1mm} \bra{#2}} % |x><y|
\newcommand{\Textbf}[1]{\hspace{3mm}\\ \textbf{#1}\\}
\newtheorem{thm}{Theorem.}[section]
\newtheorem{prop}{Proposition.}[section]


\title{Solution for "Quantum Computation and Quantum Information: 10th Anniversary Edition" by Nielsen and Chuang}
\author{goropikari}
\date{\today}

\begin{document}
\maketitle
\thispagestyle{empty}
\setcounter{page}{0} % 表紙のページを0ページにする

\section*{Copylight Notice:}
\ccbyncsa\\
This work is licensed under a \href{http://creativecommons.org/licenses/by-nc-sa/4.0/}{Creative Commons Attribution-NonCommercial-ShareAlike 4.0 International License}.


\section*{Repository}
The latest version and source \LaTeX code are located in\\ \url{https://github.com/goropikari/SolutionForQuantumComputationAndQuantumInformation}.

\section*{For readers}
This is an unofficial solution manual for "\href{http://www.cambridge.org/jp/academic/subjects/physics/quantum-physics-quantum-information-and-quantum-computation/quantum-computation-and-quantum-information-10th-anniversary-edition?format=HB&isbn=9781107002173#BBFv83H3ofgcgG3A.97}{Quantum Computation and Quantum Information: 10th Anniversary Edition}" (ISBN-13: 978-1107002173) by Michael A. Nielsen and Isaac L. Chuang.



I have studied quantum information theory as a hobby.
And I'm not a researcher.
So there is no guarantee that these solutions are correct.
Especially because I'm not good at mathematics, proofs are often wrong.
Don't trust me. Verify yourself!

If you find some mistake or have some comments, please feel free to open an issue or a PR.
\begin{flushright}
    \href{https://github.com/goropikari}{goropikari}
\end{flushright}

\tableofcontents
\newpage

%%%%%%%%%%%%%%%%%%%%%%%%%%%%%%%%%%%%%%%%%%%%%%%%%%%%%%%%%%%%%%%%%%%%%%%%%%%%%
\frontmatter
\include{chapter/errata}

\mainmatter
\input{chapter/chapter2}
\setcounter{chapter}{4}
\chapter{The quantum Fourier transform and it's application}


\Textbf{5.1} 


We can write the Fourier quantum transformation as a matrix U.
By definition, for an arbitrary ket $|x\rangle = \sum_{j=0}^{N-1} x_j | j \rangle$, the 
quantum fourier transform maps $|x\rangle$ to $|y\rangle$ as such, noting $w = e^{2 i \pi /N}$ :

$$F(|x\rangle) = |y\rangle = \frac{1}{\sqrt{N}}  \sum_{j=0}^{N-1}  \sum_{n=0}^{N-1} x_n w^{k n} | j \rangle$$

the Fourier quantum transformation can then be written as a matrix equation, giving the Fourier transform's
matrix, $F |x\rangle = |y\rangle$, with :

$$ F =  \frac{1}{\sqrt{N}} \begin{bmatrix}
	1      & 1       & \hdots & 1         \\
	1      & w       & \hdots & w^{N-1}   \\ 
	1      & w^2     & \hdots & w^{2(N-1)} \\
	\vdots & \vdots  & \vdots & \vdots     \\
	1      & w^{N-1} & \hdots & w^{(N-1)(N-1)}
\end{bmatrix}.$$

Let's now show that this matrix is indeed unitary.
If we note $A = F F^{\dagger}$, then 
$$A_{j,k} = \sum_{n=1}^N F_{j,n} F^{\dagger}_{n,k} = \frac{1}{N} (1 + w^{j-n}+w^{2(j-n)}+...+w^{(N-1)(j-n)})$$.

If $j=n$ then $A_{j,k} = \frac{N}{N} = 1$.

If $j \neq n$ then $A_{j,k} = 0$ according to the following orthogonal property over the roots of unity : 

$$ \sum_{k=0}^{N-1} (z^{jk})^\dagger z^{j' k} = 0 \quad \mathtt{if } \quad j \neq j'$$

We then see that $F F^{\dagger} = I$.
It is straightforward to show that $F F^{\dagger} =F^{\dagger} F$, we can then conclude that
the quantum Fourier transform is unitary.

\Textbf{5.2} 

the $N$ qubit state $|00...0 \rangle$ is a computational basis state, we can use directly the provided  formula 5.2 :

$$  |j\rangle \rightarrow \frac{1}{\sqrt{N}} \sum_{k=0}^{N-1}  e^{2 i \pi j k /N} |k \rangle$$

applying it to $|00...0 \rangle$ we get :

$$|00...0 \rangle \rightarrow \frac{1}{2^{N/2}} \sum_{k=0}^{2^N-1} |k \rangle$$

with $|k \rangle$ the $2^N$ computation basis states for $N$ qubits.


\setcounter{chapter}{7}
%\setcounter{chapter}{7}
\chapter{Quantum noise and quantum operations}
\Textbf{8.1}
State vector identification of the initial state is written as $\ket{\psi}$ and the evolved state is written as $U\ket{\psi} (unitary evolution). In the density operator formalism, $\ket{\psi}$ is identified with $\kb{\psi}$. Thus time development of $\rho = \kb{\psi}$ can be written by $\mathcal{E}(\rho) = U\ket{\psi} (U \ket{\psi})^\dagger = U \kb{\psi} U^\dagger = U \rho U^\dagger$.

\Textbf{8.2}
From eqn (2.147) (on page 100),
\begin{align*}
	\rho_m = \frac{M_m \rho M_m^\dagger}{\Tr (M_m^\dagger  M_m\rho ) }
					= \frac{M_m \rho M_m^\dagger}{\Tr ( M_m \rho M_m^\dagger ) }
					= \frac{\mathcal{E}_m (\rho)}{\Tr \mathcal{E}_m (\rho)}.
\end{align*}

And from eqn (2.143) (on page 99), $p(m) = \Tr (M_m^\dagger M_m \rho) = \Tr (M_m \rho M_m^\dagger) = \Tr \mathcal{E}_m (\rho)$.


\Textbf{8.3}



\Textbf{8.4}
\Textbf{8.5}
\Textbf{8.6}
\Textbf{8.7}
\Textbf{8.8}
\Textbf{8.9}
\Textbf{8.10}
\Textbf{8.11}
\Textbf{8.12}
\Textbf{8.13}
\Textbf{8.14}
\Textbf{8.15}
\Textbf{8.16}
\Textbf{8.17}
\Textbf{8.18}
\Textbf{8.19}
\Textbf{8.20}
\Textbf{8.21}
\Textbf{8.22}
\Textbf{8.23}
\Textbf{8.24}
\Textbf{8.25}
\Textbf{8.26}
\Textbf{8.27}
\Textbf{8.28}
\Textbf{8.29}
\Textbf{8.30}
\Textbf{8.31}
\Textbf{8.32}
\Textbf{8.33}
\Textbf{8.34}
\Textbf{8.35}

\input{chapter/chapter9}
\setcounter{chapter}{10}
\input{chapter/chapter11}


%%%%%%%%%%%%%%%%%%%%%%%%%%%%%%%%%%%%%%%%%%%%%%%%%%%%%%%%%%%%%%%%%%%%%%%%%%%%%

%参考文献
%\bibliographystyle{jplain}
%\bibliography{ref} % ref.bib を読み込み
\end{document}
